\documentclass{article}
\usepackage{indentfirst}
\usepackage{tabto}
\usepackage{listings}

\begin{document}
\title{The Knapsack Problem\\\large To take, or not to take}
\author{Joshua Pondrom}
\maketitle

\section{Motivation and Background}
The knapsack problem is a simple optimization problem. While the idea behind
the problem is simple, the solutions are not. The knapsack problem is a 
NP-Complete problem, which generally means the solution comes at a high cost
of computing. However, we can find a approximate solution for cheaper using
heuristics. This is good if we have less computing power, and do not care
about being accurate, as long as it is close to the optimal.

This problem has its roots in multiple fields, not only computer science. Some 
of these are combinatorics, complexity theory, cryptography, and applied
mathematics. Not only does it deal with multiple branches of study, it also has
many different applications. In general, the knapsack problem applies to any
one-dimensional constraint optimization problem. Some of these are pattern
layout, networking, scheduling, selecting investments, and more. It is also
good to note that some of these problems will be fine with the faster, 
heuristical, not-accurate algorithm, while some of the problems must have
the exact, more-expensive, algorithm.
\section{Procedures}
In this section, there will be pseudocode for both the greedy algorithm, and the
dynamic programming algorithm. Here is the dynamic programming method:
\newpage
\begin{lstlisting}
Precondition:	W is space left, w is a list of weights, v is a list of weights
		n is the value of the bag currently.
		w, and v should be non-empty, and have the same cardinality
		W should be a non-negative value
procedure Knapsack(Int:W, List:w, List:v, Int:n)
	#This is a dynamic programming solution so we store already
	# computed values in a list
	List:V
	#Initialize array to 0 where weight is 0
	for(w = 0 to W) 
		V[0][w] = 0
	for(i = 1 to n)
		for(j = 0 to W)
			if(w[i] <= j)
				V[i][j] = max(V[i-1,j], 
					     v[i] + V[i-1][j-w[i]])
			else
				V[i][j] = V[i-1, j]
	return V[n,W]
\end{lstlisting}


For the greedy solution the algorithm can either sort by the weight, taking the
smallest weights first, or sorting by value, and taking the greatest value 
first. The greedy solution is as follows:


\begin{lstlisting}
procedure Knapsack(Int:W, List:w, List:v, Int:n)
	#Sort will sort by weight ascending and done by a heapsort
	sort(w,v)
	Assert w is sorted
	while(W <= w[0], and w and n have elements remaining)
		W = W - w.pop
		n = n + v.pop
		Assert W is non-negative
		Assert w and v have 0 or more elements
	return W
\end{lstlisting}

\section{Implementation and Testing}
These can both be implemented in any language, however, I think python would be
a good fit. Python allows for focus on the algorithms without having to deal
with programming 'on the metal'. As an additional bonus, I have already
implemented heap sort in python for the previous assignment.

Invariants can be implemented by using the build in assert feature of python.
It can check correctness of the current state of the programming every time it
runs through the major loops.

Data can be fed to the problem through command line interfaces, or be generated
by the program itself by using a random command or generating data in a
predicatable way. We can time the program using the Unix tool 'time' to see
how long the program takes to process different sets of data. If needed, we can
also plot the data using python to show what happens with the data sets as the
algorithm runs.

\end{document}
